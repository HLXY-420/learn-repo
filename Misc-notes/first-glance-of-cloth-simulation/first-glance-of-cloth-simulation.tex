\documentclass{article}
\usepackage{ctex}
\usepackage{hyperref}
\usepackage{amsthm, amsmath, amssymb}
\usepackage{mathrsfs}

\title{高级布料仿真技术整理}
\author{}
\date{\today}

\begin{document}
\maketitle
\tableofcontents

\section{模拟基础}
\par 在开始之前,先简要介绍部分物理模拟领域的基础内容,作为之后提到的各种方法的基础。
\subsection{模型和模拟}
\par 模型(model)和模拟(simulation)是创建物理模拟领域的两个关键概念,模型是一组对于物体的规则或既定的定律,表征并约束物体的行为或者操作方法,而模拟则是将模型封装在一个框架之内,从而预测该模型在时间上的演变。\cite{FoPBMA}
\par 基于物理的模型也是模型的一种,它定义了一个系统如何表现各种物理规则。一般情况下,基于物理的模型是根据现实世界的物理规律构造的。但是,所有“真实”的模型实际上都是经过一定简化的——将影响较小的部分选择性忽略,专注于描述大尺度的行为和变化。
\subsection{离散化}
\par 众所周知,现实世界的时空是连续的,但是在计算机中,数据和逻辑以离散的形式存在。因此,当我们对现实世界在计算机中建模时,需要将连续的时空离散化处理。
\par 对于时间的离散化,是将时间细分为一个个时步(time step)的方法,忽略每个时步内部的变化,然后每次使用前一个状态的内容计算下一个时步的状态。而对于空间的离散化,将在下一节与布料的表示一起讨论。
\subsection{布料的表示}
\par 在计算机图形学中,几何图形通常分解为许多三角形来进行表示。单个的三角形并不亮眼,但是许多三角形组合成的三角形网格拥有强大的几何表达能力。一个三角形由三个顶点(vertices or particles)和连接它们的边(edge)构成。布料是连续的材料,但是我们会把它分解为离散的顶点表示。
\par 一个顶点一般具有坐标 $x$ 和速度 $v$ 两种内容,于是一个具有 $N$ 个顶点的服装可以用两个向量 $x \in \mathbb{R}^{3N}$ 和 $v \in \mathbb{R}^{3N}$ 表示:\cite{CloSim}
$$ x = \begin{bmatrix}
    x_{0_x} \\
    x_{0_y} \\
    x_{0_z} \\
    \vdots \\
    x_{{N-1}_x} \\
    x_{{N-1}_y} \\
    x_{{N-1}_z} \\
\end{bmatrix}, \quad\quad
v = \begin{bmatrix}
    v_{0_x} \\
    v_{0_y} \\
    v_{0_z} \\
    \vdots \\
    v_{{N-1}_x} \\
    v_{{N-1}_y} \\
    v_{{N-1}_z} \\
\end{bmatrix}.
$$
\section{质量弹簧阻尼模型}
\subsection{弹簧模型}
\subsection{显示积分法}
\subsection{隐式积分法}
\subsection{弯曲情况的改进}
\subsection{锁定问题}
\subsection{优缺点及应用}
\section{基于位置的方法}
\subsection{基于位置的动力学方法}
\subsection{投影动力学方法}
\subsection{约束动力学方法}
\section{有限元方法与弹性模型}
\subsection{有限元方法}
\subsection{有限体积方法}
\subsection{超弹性模型}
\section{物质点法}
See Papers-notes
\section{基于学习的方法}
See Papers-notes

\begin{thebibliography}{50}
    \bibitem{games103}Huamin Wang. GAMES 103 [MOOC]. GAMES-CN. \newline \href{https://games-cn.org/games103/}{https://games-cn.org/games103/}
    \bibitem{FoPBMA}House, D., \& Keyser, J.C. (2016). Foundations of Physically Based Modeling and Animation (1st ed.). A K Peters/CRC Press. \newline \href{https://doi.org/10.1201/9781315373140}{https://doi.org/10.1201/9781315373140}
    \bibitem{CloSim}Tuur Stuyck (2018). Cloth Simulation for Computer Graphics (1st ed.). Springer Cham. \newline \href{https://doi.org/10.1007/978-3-031-02597-6}{https://doi.org/10.1007/978-3-031-02597-6}
\end{thebibliography}
\end{document}