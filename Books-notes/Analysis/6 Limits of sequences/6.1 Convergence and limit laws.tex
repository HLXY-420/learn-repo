\documentclass{article}
\usepackage{amsthm, amsmath, amssymb}
\usepackage{mathrsfs}
\usepackage{indentfirst}
\usepackage{hyperref}

\title{6.1 Convergence and limit laws}
\author{Yuchen Liu}
\date{\today}

\begin{document}
\maketitle

\par In last chapter, we defined real number by formal limits: $LIM$. Then, we'll define $\lim$ and define what actually real number is via $\lim$.
\par We start from restating $\epsilon$-near \textbf{real} sequences' main structure.
\theoremstyle{definition}
\newtheorem{definition}{Definition}
\begin{definition}
    (distance between two real numbers) Given by 2 real number $x$ and $y$, we defined the distance between them $d(x,y)$ as $d(x,y) := |x-y|$.
\end{definition}
\begin{definition}
    ($\epsilon$-near real numbers) Suppose $\epsilon > 0$ is a real number, we state 2 real number $x$ and $y$ is $\epsilon$-near, if and only if $d(x,y) \leqslant \epsilon$.
\end{definition}
\par Now suppose $(a_n)_{n=m}^{\infty}$ is a sequence of real number. Then redefine Cauchy sequences the same way as before.
\begin{definition}
    (Cauchy sequences of real numbers) Suppose $\epsilon > 0$ is a real number, a sequence $(a_n)_{n=N}^{\infty}$ started from an integer $N$ can be stated \textbf{$\epsilon$-stable} if and only if to arbitrary $j,k \geq N$, $a_j$ and $a_k$ is $\epsilon$-near.
\end{definition}
...
\end{document}