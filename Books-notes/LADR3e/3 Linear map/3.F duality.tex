\documentclass{article}
\usepackage{amsthm, amsmath, amssymb}
\usepackage{mathrsfs}
\usepackage{indentfirst}
\usepackage{hyperref}

\title{3.F Duality}
\author{Yuchen Liu}
\date{\today}

\begin{document}
\maketitle

\theoremstyle{definition}
\theoremstyle{remark}
\newtheorem{definition}{Definition}
\newtheorem{example}{Example}

\section{Dual space and dual map}
\par Maps to $\textbf{F}$ play an important role in linear algebra, so they have a special name and note.
\begin{definition}
    (Linear functional) A \textbf{Linear functional} on $V$ is a map from $V$ to \textbf{$F$}, in other words, Linear maps are elements of $\mathcal{L}(V,\textbf{F})$.
\end{definition}
\begin{example}
    \textbf{Linear functional}
    \begin{itemize}
        \item Define $\phi: \textbf{R}^3 \rightarrow \textbf{R}$ as $\phi(x, y, z) = 4x - 5y + 2z$, then $\phi$ is a linear functional on $\textbf{R}^3$
        \item Pick $(c_1,\dots,c_n) \in \textbf{F}^n$, Define $\phi: \textbf{F}^n \rightarrow \textbf{F}$ as $\phi(x_i,\dots,x_n) = c_1 x_1 + \dots + c_n x_n$, then $\phi$ is a linear functional on $\textbf{F}^n$.
        \item Define $\phi: \mathcal{P}(\textbf{R}) \rightarrow \textbf{R}$ as $\phi(p) = 3p^n(5) + 7p^n(4)$, then $\phi$ is a linear functional on $\mathcal{P}(\textbf{R})$.
        \item Define $\phi: \mathcal{P}(\textbf{R}) \rightarrow \textbf{R}$ as $\phi(p) = \int_{0}^{1} p(x)\,dx $, then $\phi$ is a linear functional on $\mathcal{P}(\textbf{R})$
    \end{itemize}
\end{example}
\par And there's also a special name and note for vector space $\mathcal{L}(V, \textbf{F})$.
\begin{definition}
    (Dual space, $V'$) The vector space consisting of all linear functional on $V$ is stated as $V$'s \textbf{dual space}, noted as $V'$. In other words, $V' \in \mathcal{L}(V, \textbf{F})$
\end{definition}
\end{document}