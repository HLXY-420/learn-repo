\documentclass{article}
% usepackage{ctex}
\usepackage{amsthm, amsmath, amssymb}
\usepackage{mathrsfs}
\usepackage{cite}


\title{HOOD-report}
\author{Yuchen Liu}
\date{\today}

\begin{document}
\maketitle
\tableofcontents
\begin{abstract}
    A report after reading HOOD\cite{grigorev2023hood}. Including 3 parts: the garment representation used in HOOD, the network architecture of HOOD, and possible improvements.
\end{abstract}
\section{Innovations}
\begin{itemize}
    \item Using graph neural networks. Learning local states. Based on MeshGraphNets\cite{pfaff2021learning}.
    \item Build a message-passing scheme over a hierarchical graph that interleaves propagation steps at different levels of resolution.
    \item Using physics-based loss function which is an incremental potential for implicit time stepping.
    \item Using a simple and efficient graph coarsening strategy which allowed the network implicitly learn transition between graph levels.
\end{itemize}

\section{Garment Preprocessing}
\subsection{Build up basic graph and extensions}
\subsubsection{}
\subsection{Hierarchical graph construction}

\section{Message Passing}
\subsection{Basic message passing}
\subsection{Multi-level message passing}

\section{Physical Supervision}
\subsection{Physical garment model}
\subsection{Novel terms in the loss function}
\subsubsection{Collision term}
\subsubsection{Friction}
\subsubsection{Vertex mass and canonical geometries}
\subsection{Vertex mass and canonical geometries}
\section{Conclusion}
\subsection{Enhancement}
\begin{itemize}
    \item Computationally cheap, compared to physics-based approaches.
    \item Train once, simulate every dynamic cloth, compared to other learning-based approaches.
    \item Be able to handle changes in topology and dynamic material parameters.
    \item Truly unsupervised, without the need of ground-truth data.
    \item State-of-the-art.
\end{itemize}
\subsection{Problems remain}
\begin{itemize}
    \item Cannot simulate high speed animation, especially when body motions exceed the velocity seen at training time.
    \item Weak on solving garment self-collision.
    \item (Based on my observation) Low accuracy on prediction of inertia \& creases in motion.
    \item Batch size was locked to 1, which means low reasoning speed.
\end{itemize}

\section{Possible Improvements}
\subsection{}

\bibliographystyle{IEEEtran}
\bibliography{HOOD-notes}

\end{document}