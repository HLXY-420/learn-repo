\documentclass{article}
% usepackage{ctex}
\usepackage{amsthm, amsmath, amssymb}
\usepackage{mathrsfs}
\usepackage{cite}
\usepackage{indentfirst}
\usepackage{algpseudocodex}
\usepackage{hyperref}


\title{HOOD-report}
\author{Yuchen Liu}
\date{\today}

\begin{document}
\maketitle
\tableofcontents
\begin{abstract}
    A report after reading HOOD\cite{grigorev2023hood}. Including the analysis of its method and possible improvements.
\end{abstract}
\section{Innovations}
\begin{itemize}
    \item Using graph neural networks. Learning local states. Based on MeshGraphNets\cite{pfaff2021learning}.
    \item Build a message-passing scheme over a hierarchical graph that interleaves propagation steps at different levels of resolution.
    \item Using physics-based loss function which is an incremental potential for implicit time stepping.
    \item Using a simple and efficient graph coarsening strategy which allowed the network implicitly learn transition etween graph levels.
\end{itemize}

\section{Garment Preprocessing}
\subsection{Build up basic graph and extensions}
\par In computer graphics, a common method to model garments is to build a garment mesh of vertices and edges. Apparently, this structure is easy to translate into a graph consisting of the same vertices and edges. Moreover, the graph augments \textit{body edges} : the edge between vertices and nearest body node.
\par Feature vectors:
\begin{itemize}
    \item To nodes, $v_i$ includes velocity $\vec{v}$, normal vector $\vec{n}$, mass $\vec{m}$, node type, and local material parameters($\mu_{Lame}, \lambda_{Lame}$ and $k_{bending} $).
    \item To edges, $e_{ij}$ includes the relative position between their two nodes.
\end{itemize}
\par 
\subsection{Hierarchical graph construction}
\par In MeshGraphNets\cite{pfaff2021learning}, the simulation was implemented by the message passing through the graph. But in cloth simulation, the result is sensitive to the number of message passing steps, which makes it hard to manually set the number of steps. To address this problem, a message-passing scheme over a hierarchical graph that interleaves propagation steps at different levels of resolution was applied.
\par The hierarchical graph is a combination of several levels of coarsened garment graphs which were built by recursively applying the algorithm below:
\newpage
\begin{algorithmic}[1]
    \State \textbf{input :} fine graph $G_f(V_f,E_f)$
    \State \textbf{output:} coarse graph $G_c(V_c,E_c)$
    \Comment{$G_c \subseteq G_f, V_c \subseteq V_f, E_c \subseteq E_f$}
    \State $v_{center}$ $\gets$ center of $G_f$
    \Comment{$v_{center}$'s eccentricity is equal to the graph’s radius.}
    \For{$v_i$ in $V_f$}
        \State $d_i \gets distance(v_{center}, v_i)$
    \EndFor
    \State $V_c \gets \{v_i \in V_f | d_i \mod 2 = 0\}$
    \State $E_c \gets \{\}$
    \State $V^{interm} \gets \{v_i \in V_f | d_i \mod 2 = 1\}$
    \For{$v_i \in V^{interm}$}
        \State $V^{from} \gets \{v_j \in V_f | e_{ij} \in E_f \text{ \& } d_j = d_i - 1\}$
        \Comment{$V^{from} \subseteq V_c$}
        \State $V^{to} \gets \{v_j \in V_f | e_{ij} \in E_f \text{ \& } d_j = d_i + 1\}$
        \Comment{$V^{to} \subseteq V_c$}
        \For{$v_j$ in $V^{from}$}
            \For{$v_k$ in $V^{to}$}
                \State $push(e_{jk}, E_c)$
                \Comment{$e_{jk}$ connect nodes in $V_c$, thus $e_{jk} \in E_c$}
            \EndFor
        \EndFor
    \EndFor
\end{algorithmic}
\par Through this algorithm, we build coarsened graph $G_c$ from the input graph $G_f$, and this coarsened level is the cornerstone of \href{sec:MLMP}{Section 3.2}.
\section{Message Passing}
\par In the GNN for simulation series, the GNN is built up by MLPs which learn the relationship between edges and nodes. And the model predict next time step through $N$ message-passing steps.
\subsection{Basic message passing}
\par In each single message-passing step, edge feature are first updated as:
$$
e_{ij} \gets f_{v \rightarrow e}(e_{ij},v_{i},v_{j}) ,
$$
\par And then, node feature are updated as:
$$
v_i \gets f_{e \rightarrow v}(v_i,\sum_{j}e_{ij}^{body},\sum_{j}e_{ij}) , 
$$
\par $f_{v \rightarrow e}$ and $f_{e \rightarrow v}$ are both multi-layer perceptions.
\subsection{Multi-level message passing}
\label{sec:MLMP}

\section{Physical Supervision}
\subsection{Physical garment model}
\subsection{Novel terms in the loss function}
\subsubsection{Collision term}
\subsubsection{Friction}
\subsubsection{Vertex mass and canonical geometries}
\section{Conclusion}
\subsection{Enhancement}
\begin{itemize}
    \item Computationally cheap, compared to physics-based approaches.
    \item Train once, simulate every dynamic cloth, compared to other learning-based approaches.
    \item Be able to handle changes in topology and dynamic material parameters.
    \item Truly unsupervised, without the need of ground-truth data.
    \item State-of-the-art.
\end{itemize}
\subsection{Problems remain}
\begin{itemize}
    \item Cannot simulate high speed animation, especially when body motions exceed the velocity seen at training time.
    \item Weak on solving garment self-collision.
    \item (Based on my observation) Low accuracy on prediction of inertia \& creases in motion.
    \item Batch size was locked to 1, which means low reasoning speed.
\end{itemize}

\section{Possible Improvements}
\subsection{Details change}
\subsection{Model structure}
\subsection{"Serialization"}
\subsection{Hypergraph}

\bibliographystyle{IEEEtran}
\bibliography{HOOD-notes}

\end{document}