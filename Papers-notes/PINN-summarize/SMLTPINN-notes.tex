\documentclass{article}
\usepackage{amsthm, amsmath, amssymb}
\usepackage{mathrsfs}
\usepackage{cite}
\usepackage{indentfirst}
\usepackage{hyperref}

\title{SMLTPINN-notes}
\author{Yuchen Liu}
\date{\today}

\begin{document}
\maketitle
\tableofcontents

\begin{abstract}
    A distillation of knowledge in this long summarize. PINNs maybe a promising way to solve physical problems via NNs.
\end{abstract}
\section{What is PINNs}
\par Physics–Informed Neural Networks (PINNs) are a scientific machine learning technique used to solve problems involving Partial Differential Equations (PDEs). It approximates the solution of PDEs by training a neural network to minimize a loss function and includes terms reflecting the initial and boundary conditions along the space-time domain's boundary and the PDE residual at collocation points, selected points in the domain.
\par PINNs' input is a point in integration domain and its output is an estimated solution in that point of a DE after training. And PINNs are unsupervised.
\par PINN is not the only NN framework utilized to solve PDEs, such as the Deep Ritz method, the Galerkin method or Petrov-Galerkin method, hp-VPINN, CPINN, PCNNS.
\par Advantages:
\begin{itemize}
    \item mesh-free
    \item enable on-demand solution computation after a training stage
    \item allow solutions to be made differentiable using analytical gradients
    \item solve forward jointly and inverse problems using the same optimization problem
\end{itemize}
\section{How PINNs were built}

\subsection{Neural Network's architecture}
\subsection{Injection of Physical laws}
\subsection{Model Estimation}
\subsection{Learning Theory}

\end{document}