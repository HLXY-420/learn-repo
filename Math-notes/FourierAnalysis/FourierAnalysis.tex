\documentclass[a4paper]{article}
\usepackage{hyperref}
\usepackage{amsthm, amsmath, amssymb, amsfonts}
\usepackage{cite}

\newtheorem{prop}{Proposition}
\newtheorem{lemma}{Lemma}

\title{Notes on Fourier Analysis}
\author{HLXY}
\date{\today}

\begin{document}
\maketitle
\begin{abstract}
    Some notes after reading \textit{Fourier Analysis: An Introduction}.
\end{abstract}
\tableofcontents

\section{Preliminaries on Riemann integration}
\par In this section, we'll take a brief review on definition and main properties of Riemann integrable functions on $\mathbb{R}$ and integration of almost everywhere continuous functions.
\par Firstly, we'll give the Riemann integrable theorem on the real line. Besides the classical integration theory, we also introduce null set and give the sufficient and necessary condition where a non-continuous function is integrable.
\par Then, we'll discuss dual and multiple integrals. Especially, we'll extend the definition of integrable Schwartz function to the whole $\mathbb{R}^d$ space.

\subsection{Definition of Riemann integrable function}
\par Let $f$ be a real-value function on $[a,b]$, where $[a,b]$ is a bounded closed interval in $\mathbb{R}$. Importing the proportion $P$ to separate $[a,b]$ to finite small intervals, exactly there exists finite real numbers $x_0, x_1, \dots, x_N$, such tha
$$a=x_0 < x_1 < \dots < x_{N-1} < x_N=b,$$
For this proportion, let $I_j$ be the interval $[x_{j-1},x_j]$, $|I_j|$ be the length of $I_j$, that $|I_j|=x_j-x_{j-1}$. Define the superior and inferior Riemann sum of $f$ on proportion $P$ as
\begin{gather}
    \mathcal{U}(P, f) = \sum_{j=1}^N [\sup_{x \in I_j} f(x)]|I_j| \\
    \mathcal{I}(P, f) = \sum_{j=1}^N [\inf_{x \in I_j} f(x)]|I_j|
\end{gather}
Noting that, if $f$ is bounded, then $\mathcal{U}(P, f)$ and $\mathcal{I}(P, f)$ exist. Obviously $\mathcal{U}(P, f)$ is greater than $\mathcal{I}(P, f)$, and if for any $\epsilon>0$, there exists a proportion $P$, such that
$$\mathcal{U}(P, f) - \mathcal{I}(P, f) < \epsilon$$
then $f$ is \textbf{Riemann integrable}, or simply $f$ is integrable.
\par To define the Riemann integration of $f$, we need to make a short discussion, if proportion $P\prime$ is not a proportion of $[a,b]$, then $P^\prime$ is derived from $P$ by adding some proportion points, then we call $P^\prime$ a refinement of $P$. While adding each points, we can get
\begin{gather}
    \mathcal{U}(P^\prime, f)\leq\mathcal{U}(P, f) \\
    \mathcal{I}(P^\prime, f)\geq\mathcal{I}(P, f)
\end{gather}
Then we have, if $P_1$ and $P_2$ are two proportions of $[a,b]$, then
\begin{equation}
    \mathcal{U}(P_1, f) \geq \mathcal{U}(P_2, f) \\
\end{equation}
Take the union $P^\prime$ of $P_1$ and $P_2$, we have
\begin{equation}
    \mathcal{U}(P_1, f) \geq \mathcal{U}(P^\prime, f) \geq \mathcal{I}(P^\prime, f) \geq \mathcal{I}(P_1, f)
\end{equation}
We can get that from the boundedness of $f$,
\begin{equation}
    U=\inf_P\mathcal{U}(P, f) \text{ and } L=\sup_P\mathcal{I}(P, f)
\end{equation}
are both exist and $U \leq L$. Furthermore, if $f$ is integrable, then $U=L$, define the value of its integration as $\int_{a}^{b}f(x)dx$.
\par Lastly, for a complex function $f=u+iv$, if its real part $u$ and imaginary part $v$ are both integrable, then $f$ is integrable, and its integration is
\begin{equation}
    \int_{a}^{b}f(x)dx = \int_{a}^{b}u(x)dx + i\int_{a}^{b}v(x)dx
\end{equation}

\subsubsection{Basic Properties}
\begin{prop}
    If $f$ and $g$ are integrable on interval $[a,b]$, then:
    \begin{enumerate}
        \item $f+g$ is integrable, and $\int_{a}^{b}f(x)+g(x)dx = \int_a^b f(x)dx + \int_a^b g(x)dx$
        \item if $c \in \mathbb{C}$, then $\int_{a}^{b}c f(x)dx = c\int_a^b f(x)dx$
        \item if $f$ and $g$ are real value functions and $f(x)\leq g(x)$, then $\int_a^b f(x)dx \leq \int_a^b g(x)dx$
        \item if $c \in [a, b]$, then $\int_a^b f(x)dx = \int_a^c f(x)dx + \int_c^b f(x)dx$
    \end{enumerate}
\end{prop}
\begin{proof}[Proof]
    This proof is left to the reader as an exercise.
\end{proof}

\begin{lemma}
    If $f$ is a real-value integrable function on $[a,b]$, $\phi$ is a real-value continuous function on $\mathbb{R}$, then $\phi \circ f$ is integrable.
\end{lemma}

\end{document}